\documentclass{beamer}
\usepackage[latin1]{inputenc}
\usepackage{graphicx}
\usepackage{tipa}
\usepackage{appendixnumberbeamer}

\newcommand\Wider[2][3em]{
\makebox[\linewidth][c]{
  \begin{minipage}{\dimexpr\textwidth+#1\relax}
  \raggedright#2
  \end{minipage}
  }
}

\newcommand{\Bz}{$B_0$}

\usetheme{Madrid}

\title[BnoC Status Report]{BnoC Status Report}
\author[Tim Williams]{Tim Williams on behalf of the BnoC PWG}
\institute[Birmingham]{84th \lhcb week}
\titlegraphic{\includegraphics[height=2.1cm]{UoBLogo}}%\hspace*{2cm}\includegraphics[height=2cm]{LhcbLogo}}
\date{13/06/2017}

\usepackage{ifthen} 
\newboolean{uprightparticles}
\setboolean{uprightparticles}{false} %Set true for upright particle symbols
\usepackage{xspace} 
\usepackage{upgreek}

\input{lhcb-symbols-def}
%%%%Symbols%%%%

\def\etaorpr {\ensuremath{\Peta^{(\prime)}}\xspace}

%%%%Decays%%%%

\def\BdEtapKs    {\decay{\Bd}{\KS\etapr}\xspace}
\def\BdEtaKs     {\decay{\Bd}{\KS\Peta}\xspace}
\def\BdEtaEtapKs {\decay{\Bd}{\KS\etaorpr}\xspace}


\def\BdEtapKz    {\decay{\Bd}{\Kz\etapr}\xspace}
\def\BdEtaKz     {\decay{\Bd}{\Kz\Peta}\xspace}
\def\BdEtaEtapKz {\decay{\Bd}{\Kz\etaorpr}\xspace}

\def\LbEtapL     {\decay{\Lb}{\Lz\etapr}\xspace}
\def\LbEtaL      {\decay{\Lb}{\Lz\Peta}\xspace}
\def\LbEtaEtapL  {\decay{\Lb}{\Lz\etaorpr}\xspace}

\def\EtaGG       {\decay{\Peta}{\gamma\gamma}\xspace}
\def\EtaPiPiPi   {\decay{\Peta}{\pip\pim\piz}\xspace}
\def\EtapPiPiG   {\decay{\etapr}{\pip\pim\gamma}\xspace}
\def\EtapPiPiEta {\decay{\etapr}{\pip\pim\Peta}\xspace}


\def\BdEtapKsPiPiEta {\BdEtapKs(\EtapPiPiEta)\xspace}
\def\BdEtaKsPiPiPiz {\BdEtaKs(\EtaPiPiPi)\xspace}

\def\LbEtapLPiPiG {\LbEtapL(\EtapPiPiG)\xspace}
\def\LbEtapLPiPiEta {\LbEtapL(\EtapPiPiEta)\xspace}
\def\LbEtaLPiPiPiz {\LbEtaL(\EtaPiPiPi)\xspace}

\def\LbpkEtaz {\decay{\Lb}{\proton\kaon\etaz}\xspace}
\def\LbpkEtaorpr {\decay{\Lb}{\proton\kaon\etaorpr}\xspace}
\def\LbpkEtapr {\decay{\Lb}{\proton\kaon\etapr}\xspace}

\def\BpKetapr {\decay{\Bp}{\Kp \etapr}\xspace}

\def\LbPkPhi {\decay{\Lb}{\proton\kaon\phi}}
\def\PhiPiPiPi {\decay{\phi}{\pip\pim\piz}}

%%%%%%% PID Variables %%%%%%%
\def\probnnpi {$ProbNN_{\pi}$\xspace}
\def\probnnk {$ProbNN_{\kaon}$\xspace}
\def\probnnp {$ProbNN_{\proton}$\xspace}
\def\dllp {$\mathrm{DLL_{\pion\proton}}$\xspace}
\def\dllk {$\mathrm{DLL_{\kaon\proton}}$\xspace}




\begin{document}
%_________________________________ TITLE SLIDE ___________________________________
\begin{frame}
  \titlepage
\end{frame}

\begin{frame}{PWG Organisation}
  \begin{itemize}
  \item WG Convenors: \textbf{Stefano Perazzini, Sean Benson}
  \item Sub-WG Convenors:
    \begin{itemize}
    \item 3 Body: \textbf{Rafael Coutinho, Eli Ben-Haim}
    \item 2 \& 4 Body: \textbf{Roberta Cardinale, Laurence Carson}
    \end{itemize}
  \item Regular BiWeekly Meetings: \textbf{Thursdays 2PM}
  \end{itemize}
  \begin{center}
    \href{https://twiki.cern.ch/twiki/bin/viewauth/LHCbPhysics/BnoCTwikiPage}{\textcolor{blue}{Link to BnoC Twiki Page}}
  \end{center}
\end{frame}

\begin{frame}{PWG Liaisons}
  \begin{exampleblock}{Current Liaisons}
    \begin{itemize}
    \item Simulation: \textbf{Maria Vieites Diaz} \\
    \item Stripping: \textbf{Tim Williams (Stepping Down) \& Alvaro Gomes Dos Santos Neto} \\
    \item Trigger: \textbf{Andrea Merli} \\
    \item Tracking: \textbf{Gediminas Sarpis} \\
    \item Flavour Tagging: \textbf{Julian Garcia} \\
    \item PID: \textbf{Abhijit Mathiad} \\
    \item Calo Tools: \textbf{Jason Andrews} \\
    \item Statistics and ML Tools:\textbf{Timon Schmelzer}
    \end{itemize}
  \end{exampleblock}
  \begin{center}
    \LARGE{Thanks for all your work!}
  \end{center}
\end{frame}

\begin{frame}{BnoC Physics Case}
  \begin{columns}
    \begin{column}{.5\textwidth}
      \begin{itemize}
      \item Study decays of \bquark -hadrons to charmless hadronic final states:
        \begin{itemize}
        \item \decay{\bquark}{\uquark} tree level decays
        \item \decay{\bquark}{\squark, \dquark} penguin decays
        \end{itemize}
      \end{itemize}
    \end{column}
    \begin{column}{.5\textwidth}
      \includegraphics[width=\textwidth]{PenguinTree.jpg}
    \end{column}
  \end{columns}
    \begin{itemize}
    \item Measurements of branching fractions and CPV observables provide tests of standard model and searches for new physics
    \item \textbf{2 \& 4 Body Examples:} Searches for \decay{\PB}{baryon} decays, Angular analysis of \decay{\PB}{VV} decays, time (in)dependent CPV
    \item \textbf{3 Body Examples:} Dalitz analyses and CPV observables, Bc decays, searches for unobserved decays
    \end{itemize}
\end{frame}

\begin{frame}{BnoC Activities}
  \begin{itemize}
  \item \textbf{Currently in review:}
    6 Analyses in WG review, 7 with review committee \& 1 in collaboration wide review
  \item \textbf{Since last \lhcb week:}
    \begin{itemize}
    \item 2 Papers Published:
      \begin{itemize}
      \item $4.1\sigma$ evidence for the decay \decay{\Bp}{\proton \Lbar} (LHCb-PAPER-2016-048)\\
      \item Upper limit on \BF(\decay{\Bs}{\phiz \etapr}) $< 0.82 \times 10^{-6}$ at $90\%$ CL. (LHCB-PAPER-2016-060)
      \end{itemize}
    \item 2 Papers Submitted:
      \begin{itemize}
      \item First observation of a baryonic \Bs decay (LHCB-PAPER-2017-012)
      \item Observation of the charmless baryonic decays \decay{\PB^{0}_{(s)}}{\proton \antiproton h^{+} h^{-}} (LHCb-PAPER-2017-005)
      \end{itemize}
    \end{itemize}
  \item 32 Analyses in preperation - only able to present a few today.
  \item Details of all past and present analyses can be found in the \href{https://lhcb-wg.web.cern.ch/lhcb-WG/bnoc/list.py}{\textcolor{blue}{WG database}}
  \end{itemize}
\end{frame}

\begin{frame}
  \begin{center}
    \begin{block}{}
      \centering \Huge \decay{\PB^{0}_{(s)}}{\proton\antiproton} Update
    \end{block}
    \href{https://lhcb-wg.web.cern.ch/lhcb-WG/bnoc/listentry.py?name=2011\%2B12+B+-\%3E+p+pbar+BFs\&cat=analysis}{\textcolor{blue}{Link to WG Database}}
  \end{center}
\end{frame}

\begin{frame}{\decay{\PB^{0}_{(s)}}{\proton\antiproton} Motivation and Status}
  \begin{block}{Motivation}
    \begin{itemize}
    \item No baryonic 2 body charmless $\PB^{0}$ decay has been observed.
    \item \decay{\PB^{0}_{(s)}}{\proton\antiproton} decays predicted to be simplest to search for experimentally
    \end{itemize}
  \end{block}
  \begin{itemize}
  \item Previous analysis using 2011 data saw $3.3\sigma$ evidence for \decay{\Bd}{\proton\antiproton} but no evidence for the suppressed \decay{\Bs}{\proton\antiproton} seen.
  \item Hopefully addition of 2012 data can lead to an observation.
  \end{itemize}
  \begin{columns}
    \begin{column}{0.65\textwidth}
      \begin{itemize}
      \item Branching fraction results using 2011 data: \\
        \BF(\decay{\Bd}{\proton\antiproton})$ = (1.47^{+0.62+0.35}_{-0.51-0.14})\times 10^{-8}$\\
        \BF(\decay{\Bs}{\proton \antiproton})$ = (2.84^{+2.03 +0.85}_{-1.68 -0.18}) \times 10^{-8}$
      \item All previous theory calculations ruled out by at least 1 order of magnitude!
      \end{itemize}
    \end{column}
    \begin{column}{0.34\textwidth}
      \includegraphics[width=\textwidth]{2011PPbar.pdf}
    \end{column}
  \end{columns}
\end{frame}

\begin{frame}{Selection}
  \small
  \begin{itemize}
  \item PID Selection optimises \dllppi and \dllpk cuts for Punzi FoM with $a=5\sigma$
  \item MVA selection (applied after PID selection) makes use of MLP with 10 variables, again optimised for Punzi FoM.
    \begin{center}
      \includegraphics[width=.4\textwidth]{PPOptimisation.pdf}
      \includegraphics[width=.4\textwidth]{PPBarPIDOptimisation.pdf}
    \end{center}
  \item Normalisation channel, \decay{\Bd}{\Kp\pim}, PID selection optimised for maximum selection efficiency whilst rejecting various mis-ID backgrounds.    % READ.
  \item Similar MLP also used in normalisation channel, but optimised for signal significance.
  \end{itemize}
\end{frame}

\begin{frame}{Fits}
  \small
  \begin{itemize}
  \item \textbf{Normalisation channel:} Mass fit seperated by charge due to known production assymmetries
  \begin{center}
    \includegraphics[width=0.32\textwidth]{PPNormMassFit.pdf}
    \includegraphics[width=0.32\textwidth]{PPNormMassFitBar.pdf}
  \end{center}
  \item \textbf{Signal channel:} Only signal and combinatorial background present - $38.7\pm8.4(stat only)$ \decay{\Bd}{\proton\antiproton} decays observed!
    % \item $N(\decay{\Bd}{\proton\antiproton} = 38.7\pm8.4 (stat only)$ and $N(\decay{\Bs}{\proton\antiproton} = 1.5\pm4.4 (stat only)$.
    \begin{center}
      \includegraphics[width=.4\textwidth]{PPSignalMassFit.pdf}
    \end{center}
  \end{itemize}
\end{frame}

\begin{frame}{Results and Outlook}
  \begin{itemize}
  \item First obeservation of \decay{\Bd}{\proton \antiproton} with significance of $5.3\sigma$ -first obseration of 2 body charmless baryonic \Bd decay!
  \item \BF(\decay{\Bd}{\proton \antiproton})$ = (1.25\pm0.27\pm0.18) \times 10^{-8}$.
  \item \decay{\Bs}{\proton \antiproton} yield of only $1.5\pm4.4$ events, limit set using Feldman-Cousins method.
    \begin{center}
      \includegraphics[width=0.4\textwidth]{PPBarFC.pdf}
    \end{center}
  \item \BF(\decay{\Bs}{\proton \antiproton})$ < 1.5 \times 10^{-8} $ at $90\%$ confidence level.
  \item Analysis close to publication, paper currently in collaboration wide review.
  \end{itemize}
\end{frame}

\begin{frame}
  \begin{block}{}
    \centering Amplitude analysis of \Large \decay{\Bd}{(\pip \pim)(\Kp\pim)} decays
  \end{block}
  \centering \href{https://lhcb-wg.web.cern.ch/lhcb-WG/bnoc/listentry.py?name=2011\%2B12+B+-\%3E+K\%2A0+rho0+BF\&cat=analysis}{\textcolor{blue}{Link to WG database}}
\end{frame}

\begin{frame}{The \decay{\Bd}{\rhoz(\pi \pi)\Kstarz(\Kp\pim)} decay}
  \begin{columns}
    \begin{column}{.75\textwidth}
      \begin{itemize}
        \item Proceeds either via:
        \begin{itemize}
        \item A doubly Cabibbo suppressed tree
        \item A gluonic \decay{\bquark}{\squark} penguin
        \item Both have similar amplitudes, good for $a_{cp}$!
        \end{itemize}
      \item Self tagged decay: \\ \decay{\textcolor{red}{\Bd}}{\rhoz(\pi \pi)\Kstarz(\textcolor{red}{\Kp}\pim)} \\ \decay{\textcolor{red}{\Bdb}}{\rhoz(\pi \pi)\Kstarz(\textcolor{red}{\Km}\pip)}
      \item \textbf{Aim:} Measure longitudinal polarisation fraction $f_L$, \ACP and accessible triple product assymmetries.
      \item Amplitude analysis required to disentangle different helicity amplitudes and final states.
      \end{itemize}
    \end{column}
    \begin{column}{.24\textwidth}
      \includegraphics[width=\textwidth]{KstarRhoFeynman.pdf}
    \end{column}
  \end{columns}
\end{frame}

\begin{frame}{Event Selection}
  \begin{itemize}
  \item A BDT is used to reject combinatorial background, cut chosen rejects $~99\%$ of background samples whilst retaining $\geq90\%$ of signal MC.
  \item Charm resonances removed with vetoes
  \item Tight PID cuts remove most mis-ID backgrounds. \\ \vspace{0.2cm}
    \begin{columns}
      \begin{column}{0.6\textwidth}
        \textbf{\decay{\Bs}{\Kstarz\Kstarzb} background:}
        \begin{itemize}
        \item PID cuts modified to create seperate sample with high purity of these events, M(\kaon\pion\kaon\pion) fitted to determine level of contamination.
        \item Simulated \decay{\Bs}{\Kstarz\Kstarzb} events injected with negative weights to remove \decay{\Bs}{\Kstarz(\Km\pip)\Kstarzb(\Kp\pim)} background.
        \end{itemize}
      \end{column}
      \begin{column}{0.39\textwidth}
        \includegraphics[width=\textwidth]{KstarKstarSelection.pdf}
      \end{column}
    \end{columns}
  \end{itemize}
\end{frame}

\begin{frame}{Four Body Mass Fit}
  \begin{itemize}
  \item \textbf{Aim: Perform fit to M(\pion\pion\kaon\pion) distribution and extract sWeights}
  \item \decay{\PB^{0}_{(s)}}{(\Kp\pim)(\pip\pim)} peaks modelled with Hypatia function, combinatorial background modelled with exponential.
    \includegraphics[width=.9\textwidth]{KstarRho4Bod.pdf}
  \item Total of 11290 \decay{\Bd}{(\pip\pim)(\Kp\pim)} signal events.
  \end{itemize}
\end{frame}

\begin{frame}{Amplitude Analysis Formalism}
  \begin{itemize}
  \item Isobar model used to paramaterise the total \decay{\Bd}{(\pip\pim)(\Kp\pim)} decay amplitude:
    \begin{equation}
      A_{T}=\sum_i A_i \times g_i(\theta_{\pip\pim},\theta_{\Kp\pim},\phi) \times M_i(m_{\pip\pim},m_{\Kp\pim})
    \end{equation}
  \item $g_i$ are the legendre polynomials describing the angular distributions
  \item $M_i(\pip\pim,\Kp\pim)$ are functions to model the effective two body masses.
  \end{itemize}
  \includegraphics[width=.5\textwidth]{KstarRhozAngles.pdf}
\end{frame}

\begin{frame}{Amplitude Fit}
  \scriptsize
  \begin{itemize}
  \item A PDF has been created to model 11 decay channels and 14 amplitudes.
  \end{itemize}
  \begin{columns}
    \begin{column}{0.3\textwidth}
      \textbf{Nominal Mass Propogators:}
      \begin{itemize}
      \item \textcolor{red}{\Kstarz, \Pomega, $f_0(500)$,$f_0(1370)$:}\\
        Relativistic Breit-Wigner
      \item \textcolor{red}{\rhoz:}\\
        Gounaris-Sakurai
      \item \textcolor{red}{$f_0(980)$:}\\
        Flatte
      \item \textcolor{red}{$(\kaon\pion)_0$}\\
        LASS
      \end{itemize}
    \end{column}
    \begin{column}{0.69\textwidth}
      \includegraphics[width=\textwidth]{KstarRhoAmplitudes.pdf}
    \end{column}
  \end{columns}
  \begin{itemize}
  \item Detector acceptance accounted for with normalisation weights
  \end{itemize}
\end{frame}

\begin{frame}{Results and Outlook}
  \includegraphics[width=.75\textwidth]{FiveBodyFitResults.pdf}
  \begin{itemize}
  \item All CPV observables still blind
  \item Analysis in WG review.
  \end{itemize}
\end{frame}

\begin{frame}
  \begin{block}{}
    \centering \Huge Search for the decay  \LbpkEtapr
  \end{block}
  \centering \href{https://lhcb-wg.web.cern.ch/lhcb-WG/bnoc/listentry.py?name=2011\%2B12+Lb+-\%3E+p+K+eta\%28\%27\%29+BFs\&cat=analysis}{\textcolor{blue}{Link to WG database}}
\end{frame}

\begin{frame}{Motivation}
  \begin{itemize}
  \item \etapr still not fully understood - size of gluon component in meson wavefunction?
    \begin{equation}
      \ket{\etapr} \approx \cos{\phi_{g}}\sin{\phi_{p}}\frac{1}{\sqrt{2}}\ket{\uubar + \ddbar} + \cos{\phi_{g}}\cos{\phi_{p}}\ket{\ssbar}+sin{\phi_g}\ket{gg}
    \end{equation}
    Latest measurement by \lhcb:  $\phi_{g}=(0\pm 24.6)\degrees$ (arXiv:1411.0943),  $\phi_{p} = (43.5^{+1.4}_{-1.3}) \degrees$.
  \item The decay of a \bquark baryon to an \etaorpr final state has never been observed, completely unexplored area of charmless B physics.
  \item $3\sigma$ evidence seen for \LbEtaL at \lhcb, $\mathcal{B}  = 9.3^{+7.3}_{-5.3} \times 10^{-6}$, and limit set on \LbEtapL, $\mathcal{B}<3.1 \times 10^{-6}$ (90\%) (LHCB-PAPER-2015-019).
  \end{itemize}
\end{frame}

\begin{frame}{Analysis Strategy}
  \begin{itemize}
  \item Perform blind search for \LbpkEtapr using Run I data.
  \item Reconstruct \etapr in two channels: \EtapPiPiG (BF=0.291) and \EtapPiPiEta (\EtaGG) (BF=0.169)
  \item \BpKetapr (\EtapPiPiG) used as a control channel for both rare channels.
    \begin{block}{Ratio of branching fractions extracted as:}
      \begin{equation*}
        \frac{\mathcal{B}(\Lb)}{\mathcal{B}(\Bp)} = \big(\frac{\epsilon_CN_\gamma}{\epsilon_\gamma N_C} + \frac{\epsilon_C N_\eta}{\epsilon_\eta N_C}\big)\frac{\mathcal{B}(\EtapPiPiG)}{\mathcal{B}(\EtapPiPiEta) + \mathcal{B}(\EtapPiPiG)} \frac{f_u}{f_\Lambda}
      \end{equation*}
    \end{block}
  \item $N_C (\epsilon_C)$,$N_{\gamma} (\epsilon_{\gamma})$ and $ N_{\eta} (\epsilon_{\eta})$ are yields (efficiencies) in control, \EtapPiPiG and \EtapPiPiEta channels respecitvely.
  \item Fit all channels simultaneously and extract ratio of branching fractions directly from fit.
  \end{itemize}
\end{frame}

\begin{frame}{Selection}
  \small
  \begin{itemize}
  \item Train seperate BDTs for each rare channel and optimise for Punzi FoM with $\sigma=5$.
    \begin{center}
      \includegraphics[width=.41\textwidth]{2012IOPpipig.pdf}
      \includegraphics[width=.41\textwidth]{pipieta12IOP.pdf}
    \end{center}
  \item Similar BDT used for normalisation channel but optimised for signal significance.
  \item 3D PID optimisation performed, results in tight cuts on \texttt{ProbNN} PID variables for Proton and Kaon, looser cuts for Pions.
  \item Specific mass vetoes for charm resonances and \decay{\Lb}{\proton \Km \pip \pim} decays.
%  \item Require M(\pip \pim)\textgreater 510.0\mev in \EtapPiPiG channel to remove low mass background
%  \item Require $480.0\mev < M(\etaz)_{DTF}< 620.0 \mev$ in \EtapPiPiEta channel.

  \end{itemize}
\end{frame}

\begin{frame}{Efficiency Calculation Procedure}
\begin{itemize}
\item Rich resonant structure expected in M(\proton \Km) spectrum but not a priori known.
  \begin{columns}
    \column{.49\textwidth}
    \begin{block}{if Significance \textgreater $5\sigma$}
      \begin{itemize}
      \item Bin efficiency in square Dalitz plot (SDP) variables m' and $\theta'$.
      \item Extract sWeights from mass fit - background subtraction.
      \item Efficiency taken as:
        \begin{equation}
          \epsilon = \frac{\sum_i^{N}W_i}{\sum_i^{N}\frac{W_i}{\epsilon_i}}
        \end{equation}
        where $\epsilon_i$ is efficiency in the bin of SDP occupied by event i.
      \end{itemize}
    \end{block}
    \column{.49\textwidth}
    \begin{itemize}
    \item Need to take into account efficiency variation over phase space of \LbpkEtapr decay.
    \end{itemize}
    \begin{block}{if Significance \textless $5\sigma$}
      \begin{itemize}
      \item Fill 1D histogram of efficiency for each bin of SDP.
      \item Use mean value for efficiency and assign RMS as systematic uncertainty.
      \end{itemize}
    \end{block}
  \end{columns}
\end{itemize}
\end{frame}

\begin{frame}{Signal Channel Mass Fits}
  \begin{itemize}
  \item All signal shapes modelled with DCB
  \item Combinatorial background modelled with exponential (2nd order polynomial) in rare (control) channel.
  \item \EtapPiPiG channel also suffers from \decay{\Lb}{4h + \piz} background, modelled with bifurcated Gaussian - shape fixed to MC.
  \end{itemize}
  \includegraphics[width=.49\textwidth]{PiPiGFit.pdf} \includegraphics[width=.49\textwidth]{PiPiEta.pdf}
\end{frame}

\begin{frame}{Expected Sensitivity and Outlook}
  \begin{itemize}
  \item Toy studies have been performed for a range of \BF \xspace assumptions - statistical significance caluclated with Wilk's theorem.
    \begin{center}
      \includegraphics[width=.6\textwidth]{LbpketaprSensitivity.pdf}
    \end{center}
  \item $~77\%$ chance of $5\sigma$ observation if \BF(\LbpkEtapr)$=4\times10^{-6}$
  \item Analysis in WG review.
  \end{itemize}
\end{frame}

\begin{frame}
  \begin{block}{}
    \centering \Huge Search for CP violation in \decay{\PXi_{b}}{\proton \Kp \Km} decays.
  \end{block}
\end{frame}

\begin{frame}{Overview}
  \begin{itemize}
  \item \decay{\PXi_{b}}{\proton \Kp \Km} was observed for the first time using Run I data
  \item Next step is a search for CPV using amplitude analysis - $\geq \sim300$ events required.
    \begin{center}\includegraphics[width=0.6\textwidth]{XiBFit.pdf}\end{center}
  \item Plan to achieve this - Improve selection and add Run II data.
  \end{itemize}
\end{frame}

\begin{frame}{\decay{\PXi_{b}}{\proton \Kp \Km} Run I improvements}
  \small
  \begin{block}{Changes to Run I selection}
    \begin{itemize}
    \item Switch to using inclusive \texttt{Xb2phhLine} stripping line - $15\%$ improvement in stripping efficiency.
    \item Include PID in MVA, use \texttt{xGBoost} algorithm and optimise for signal significance.
    \end{itemize}
  \end{block}
  \begin{columns}
    \begin{column}{.5\textwidth}
      \begin{itemize}
      \item Very preliminary fit to Run I data shows yield of $\sim 140$ events with new selection.
      \item Should be able to observe $\geq \sim300$ signal events when Run II data is included.
      \end{itemize}
    \end{column}
    \begin{column}{.5\textwidth}
      \includegraphics[width=\textwidth]{Run1ImprovedXib.pdf}
    \end{column}
  \end{columns}
\end{frame}

\begin{frame}{CPV in \decay{\Lb}{\proton h h h}}
  \begin{block}{}
    \begin{center}
      \Huge CPV in \decay{\Lb}{\proton h h h}
    \end{center}
  \end{block}
\end{frame}

\begin{frame}{Current Status}
  \small
  \begin{columns}
    \begin{column}{.4\textwidth}
      \includegraphics[width=\textwidth]{Run1CPV.pdf}
    \end{column}
    \begin{column}{.65\textwidth}
      \begin{itemize}
      \item Using Run I data $3.3\sigma$ evidence for CPV in \decay{\Lb}{\proton \pim\pip\pim} decays was observed.
      \item Used T-odd triple product asymmsetries.
      \item First evidence for CPV in a baryon decay! \href{http://www.nature.com/nphys/journal/v13/n4/full/nphys4021.html}{\textcolor{blue}{Nature Physics 13 (2017)}}
      \item More results from other deacy modes in review comittee!
      \end{itemize}
    \end{column}
  \end{columns}
  \begin{center}\includegraphics[width=.75\textwidth]{NewCPVChannels.pdf}\end{center}
\end{frame}

\begin{frame}{Run II Update}
  \small
  \begin{itemize}
  \item Run II update of CPV analyses planned.
  \item Expect factor $\sim3$ increase in signal yield with addition of 2016 and 2017 data
  \begin{center}
    \includegraphics[width=.8\textwidth]{Run2FirstLook.pdf}
  \end{center}
  \item First look at 2016 MagDown data shows promising yield - selection not optimised and cross-feed studies needed
  \end{itemize}
\end{frame}

\begin{frame}{Energy Test}%jax2dtki
  \begin{itemize}
  \item Work under way to use energy test method to search for both P even and P odd CP violation - \href{https://arxiv.org/abs/1612.04705}{\textcolor{blue}{arxiv:1612.04705}}
  \item Define test statistic based on distances between particles and anti-particles in phase space.
  \end{itemize}
  \includegraphics[width=.5\textwidth]{TestSta.pdf}\hspace{0.5cm}\includegraphics[width=.4\textwidth]{DistanceFunc.pdf}
  \begin{columns}
    \begin{column}{.49\textwidth}
      \begin{itemize}
      \item Sensitivity studies based on RunI + 2015 +2016 data performed using cocktail of RapidSim samples, CPV artificially introduced by increasing relative fraction of $\PDelta^{+}$ events by $30\%$.
      \end{itemize}
    \end{column}
    \begin{column}{.49\textwidth}
      \includegraphics[width=\textwidth]{SPVSensitivity.pdf}
    \end{column}
  \end{columns}
\end{frame}

\begin{frame}{Summary of \decay{\Lb}{\proton h h h} CPV searches}
  \begin{itemize}
  \item First evidence for CP violation has been seen in \decay{\Lb}{\proton \pim \pip \pim} decays at level of $3.3\sigma$ using Run I data - Run II update planned.
  \item Run I results for \decay{\Lb}{\proton \Km \pip \pim}, \decay{\Lb}{\proton \Km \Kp \Km} and \decay{\Lb}{\proton \pip \Km \Km} decays still blind and in review comittee
  \item Work started on using energy test method to search for CP violation in \decay{\Lb}{\proton \pim \pip \pim} decays using Run I+ Run II data.
  \end{itemize}
\end{frame}

\begin{frame}{Conclusions}
  \begin{itemize}
  \item Lots of interesting work taking place in BnoC group on wide range of analyses.
  \item Several analyses in review using Run I data
  \item Many more analyses using Run II data in preperation: e.g \decay{\Bs}{\phiz \phiz} (in WG review), \decay{\Bc}{\Kp\Km\pip}, \decay{\Bs}{\KS \KS}, TD CPV in \decay{\PB^{0}_{(s)}}{hh} decays.
  \item Lookout for dedicated talks on \decay{\Bp}{\pip\pim\pip} (Thursday) and \decay{\PB^{0}_{(s)}}{\KS h h} (Today).
  \end{itemize}
\end{frame}
  
\appendix
\begin{frame}{Backup}
  Backup
\end{frame}

\end{document}
